

%\VignetteDepends{harbChIP}
%\VignetteIndexEntry{upstream sequence management for yeast}
%\VignetteKeywords{yeast, sequence analysis}


\documentclass[12pt]{article}
\usepackage{amsmath,pstricks,fullpage}
\usepackage[authoryear,round]{natbib}
\usepackage{hyperref}
\usepackage{theorem}
\usepackage{float}
\usepackage{ifthen}

\newcommand{\scscst}{\scriptscriptstyle}
\newcommand{\scst}{\scriptstyle}

\newcommand{\R}{{\textsf{R}}}
\newcommand{\code}[1]{{\texttt{#1}}}
\newcommand{\term}[1]{{\emph{#1}}}
\newcommand{\Rpackage}[1]{\textsf{#1}}
\newcommand{\Rfunction}[1]{\texttt{#1}}
\newcommand{\Robject}[1]{\texttt{#1}}
\newcommand{\Rclass}[1]{{\textit{#1}}}
\newcommand{\Rmethod}[1]{{\textit{#1}}}
\newcommand{\Rfunarg}[1]{{\textit{#1}}}

\usepackage{/Users/stvjc/Work/ExternalSoft/R27/share/texmf/Sweave}
\begin{document}

\title{The sceUpstr support in package harbChIP}
\author{HJB/VJC}
\maketitle

\tableofcontents

\section{Introduction}

The intent of this package is to allow code like the following:
\begin{Schunk}
\begin{Sinput}
> library(harbChIP)
> data(sceUpstr)
> sceUpstr
\end{Sinput}
\begin{Soutput}
upstreamSeqs instance, organism  sce 
There are  6674  entries
first keys: 
[1] "YAL001C" "YAL002W" "YAL003W" "YAL004W" "YAL005C"
\end{Soutput}
\begin{Sinput}
> getUpstream("YAL001C", sceUpstr)
\end{Sinput}
\begin{Soutput}
$YAL001C
  500-letter "DNAString" instance
Value: CTGTACCACTATAATAATTTATCTTGATCGTATTA...GACGTTTGGTTGAAGCCAACTAGCCACAAGAAAA
\end{Soutput}
\end{Schunk}

\section{Building the sceUpstr object}

Upstream sequences of length 500bp were obtained from the SGD website:\\
\texttt{www.yeastgenome.org -> Download Data -> FTP}\\
\texttt{sequence/genomic\_sequence/orf\_dna/archive/utr5\_sc\_500.20040206.fasta.gz}

\begin{Schunk}
\begin{Sinput}
> fname = system.file("extdata/utr5_sc_500_20040206.fasta", package = "sceUpstr")
> utr5 = readFASTA(fname)
> sceUpstr = buildUpstreamSeqs2(utr5)
> save(sceUpstr, file = "sceUpstr.rda")
\end{Sinput}
\end{Schunk}

\section{Checking a finding of Harbison et al.}

It is asserted in Fig 1 B of the paper that GGCGCTA is specifically bound by Snt2.
We will examine the frequency of this heptamer in upstream regions and relate to the
binding ratio distribution.

\end{document}
